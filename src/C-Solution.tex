% ==================================================
% @brief    模型的建立与求解
% ==================================================


\mcmSection{模型的建立与求解}

\mcmSubsection{问题1:通过解三角行求出多波束探测的覆盖宽度}

\mcmSubsubsection{覆盖宽度公式}

在问题1中,$\alpha$是测垂面与海底坡面的交线与水平面的夹角,实际上正是测坡角。也就是说,在问题一中,存在等式:

\begin{equation}
    \gamma = \alpha
\end{equation}

\begin{figure}[h]
    \centering
    \includegraphics[scale=0.4]{res/img/探测船探测正视图.png}
    \caption{探测船探测正视图}
    \label{fig:探测船探测正视图}
\end{figure}

根据图\ref{fig:探测船探测正视图},为了求出覆盖宽度,可以抽象出如下的几何问题:

已知$\vartriangle PAB$是以$PA$、$PB$为腰的等腰三角形,其中,点$C$是$BP$上任意一点, PM是AB的垂线,且$\angle APB = \theta$, $\angle CAB = \gamma$, $PM = D$(如图\ref{fig:理想情况下的覆盖区域几何图}),求出$AC$在$AB$上的投影。

\begin{figure}[h]
    \centering
    \includegraphics[scale=0.4]{res/img/理想情况下的覆盖区域几何图.png}
    \caption{\href{https://www.geogebra.org/m/xcvstdzg}{\textcolor{blue}{理想情况下的覆盖区域几何图}}}
    \label{fig:理想情况下的覆盖区域几何图}
\end{figure}

由题意可设$AM=x_1$,$MC=x_2$,$\angle PAM = \frac{\pi - \theta}{2} - \gamma = \varphi $.

对于$\vartriangle PAM$,根据正弦定理,存在方程:

\begin{equation*}
    \frac{\sin\angle APM}{AM} = \frac{\sin\angle PAM}{PM}
\end{equation*}

对于$\vartriangle PCM$,根据正弦定理,存在方程:

\begin{equation*}
    \frac{\sin\angle CPM}{CM} = \frac{\sin\angle PCM}{PM}
\end{equation*}

综上所述,将对应的值带入方程,可得下列方程组:

\begin{equation}
    \begin{cases}
        \varphi = \frac{\pi - \theta}{2} - \gamma                 \\
        \frac{\sin \frac{\theta}{2}}{x_1} = \frac{\sin\varphi}{D} \\
        \frac{\sin \frac{\theta}{2}}{x_2} = \frac{\sin(\pi-\theta-\varphi)}{D}
    \end{cases}
\end{equation}

解上述方程组以后,可得:

\begin{equation}
    \begin{aligned}
        AC & = x_1 + x_2                                                                                                           \\
           & = \frac{\sin\frac{\theta}{2}}{\sin\varphi}D + \frac{\sin\frac{\theta}{2}}{\sin(\varphi + \theta)}D                    \\
           & = D\sin\frac{\theta}{2}\left(\frac{1}{\cos(\frac{\theta}{2}+\gamma)} + \frac{1}{\cos(\frac{\theta}{2}-\gamma)}\right)
    \end{aligned}
\end{equation}

综上,可得如下结论:

\begin{mcmTheorem}{覆盖宽度公式}
    测量船的覆盖宽度$W$,与它的多波束换能器的开角$\theta$、测坡角$\gamma$、距离海底的深度$D$有关,满足如下等式:

    \begin{equation}
        W =
        Proj AC =
        D\sin\frac{\theta}{2}\left(\frac{1}{\cos(\frac{\theta}{2}+\gamma)} + \frac{1}{\cos(\frac{\theta}{2}-\gamma)}\right) \cdot \cos \gamma
        % \label{equ:覆盖宽度公式}
    \end{equation}
\end{mcmTheorem}

\mcmSubsubsection{海底深度公式}

在问题一的情景中,当测量船与中心位置有一定偏移时,深度会发生改变,并且变化量$\Delta D$如图\ref{fig:深度变化}所示。



\begin{figure}[h]
    \centering
    \includegraphics[scale=0.5]{res/img/深度变化.png}
    \caption{\href{https://www.geogebra.org/m/ytasr4g4}{\textcolor{blue}{深度变化}}}
    \label{fig:深度变化}
\end{figure}

其中,线段$FC$是海平面,平行于线段$AB$,点$E$处是中心位置。为方便描述,设$\overrightarrow{FC}$、$\overrightarrow{EH}$分别为$x$轴、$y$轴正方向。

根据几何关系,易得:

\begin{equation}
    \Delta D = d\tan\alpha
\end{equation}

故中心距离\newline\newline\newline\newline

\begin{mcmTheorem}{海底深度函数}
    以海底深度由深到浅为正方向,确定某个中心位置$O$后,某个位置的深度$D(x)$与坡面坡度$\alpha$、中心位置的深度$D(0)$、与中心的距离$d$有关,满足如下公式:

    \begin{equation}
        D(d) = D(0) + d\tan\alpha
    \end{equation}
\end{mcmTheorem}

\mcmSubsubsection{总结}

借原题中测线相互平行且海底地形平坦时的重叠率定义:

\begin{equation}
    \eta = 1 - \frac{d}{W}
\end{equation}

其中$d$为相邻两条测线的间距,$W$为条带的覆盖宽度。

在本小节中,当测线方向与海底坡面的法向在水平面上投影垂直时,重叠率为:

\begin{equation}
    \eta = 1-\frac{2d}{S_W}
\end{equation}

其中$d$为相邻两条测线的间距,$S_W$为相邻两条条带的覆盖宽度之和。

综上所述,存在方程组:

\begin{equation}
    \begin{cases}
        W = Proj AC = D \sin\frac{\theta}{2}\left[
            \frac{1}{\cos(\frac{\theta}{2}+\gamma)} +
            \frac{1}{\cos(\frac{\theta}{2}-\gamma)}
        \right] \cos\gamma \\
        D(d) = D(0) + d\tan\alpha \\
        \eta = 1 - \frac{2d}{S_w}
    \end{cases}
\end{equation}

其中,$\theta$为多波束换能器的开角,$D(0)$为中心海域的深度,$S_w$为覆盖宽度

当$\theta = 120^\circ$,$\gamma=\alpha=1.5^\circ$,海域中心点处的海水深度$D_5 = 70m$时,带入上述模型,计算可以得出结果:

% Please add the following required packages to your document preamble:
% \usepackage{booktabs}
\begin{table}[h]
    \centering
    \caption{\textbf{问题1的计算结果(保留两位小数)}}
    \scalebox{0.7}{
        \begin{tabular}{cccccccccc}
            \toprule
            \begin{tabular}[c]{@{}l@{}}测线距中心点处的距离/$m$\end{tabular}  & $-800$   & $-600$   & $-400$   & $-200$   & $0$      & $200$    & $400$    & $600$    & $800$    \\ \midrule
            海水深度/$m$                                                      & $90.95$  & $85.71$  & $80.47$  & $75.24$  & $70.00$  & $64.76$  & $59.53$  & $54.29$  & $49.05$  \\
            覆盖宽度/$m$                                                      & $315.71$ & $297.53$ & $279.35$ & $261.17$ & $242.99$ & $224.81$ & $206.63$ & $188.45$ & $170.27$ \\
            \begin{tabular}[c]{@{}l@{}}与前一条测线的重叠率/$\%$\end{tabular} & -        & $34.77$  & $30.66$  & $26.00$  & $20.66$  & $14.49$  & $7.29$   & $-1.25$  & $-11.51$ \\ \bottomrule
        \end{tabular}
    }
\end{table}


\mcmSubsection{问题二:一般情况下,重叠率求解}

根据题意,可以抽象出如下的几何问题:

有两个直三棱柱如图\ref{fig:一般情况下的覆盖区域几何图}所示摆放,三棱柱$BEC-AFD$的侧边与三棱柱$DHC-FGE$的侧边相重合,其中,$AF$垂直于$FE$,$FG$垂直于$GE$,$\angle AFG=\beta$,$\angle CBE=\alpha$。设平面$ABCD$与平面$HCEG$的交线为$l_s$(图中未画出),求$l_s$与平面$ABEF$的夹角。

\begin{figure}[h]
    \centering
    \includegraphics[scale=0.4]{res/img/一般情况下的覆盖区域几何图.png}
    \caption{\href{https://www.geogebra.org/m/absuxwpk}{\textcolor{blue}{一般情况下的覆盖区域几何图}}}
    \label{fig:一般情况下的覆盖区域几何图}
\end{figure}

分别设平面ABCD与平面HCEG的法向量为$\tau_\text{坡}$和$\tau_\text{测}$,$l_s$的方向向量为$v_s = (x_0, y_0, z_0)$.

根据平面法向量的性质,存在方程组:

\begin{equation}
    \begin{cases}
        \tau_\text{坡} \cdot v_s = 0 \\
        \tau_\text{测} \cdot v_s = 0
    \end{cases}
\end{equation}

又因为$\angle AFG=\beta$,$\angle CBE=\alpha$,可得:

\begin{equation}
    \begin{cases}
        \tau_\text{坡} = (\sin\alpha, 0, \cos\alpha) \\
        \tau_\text{测} = (\cos\beta, \sin\beta, 0)
    \end{cases}
\end{equation}

故有方程组:

\begin{equation}
    \begin{cases}
        x_0\sin\alpha + z_0\cos\alpha = 0 \\
        x_0\cos\beta + y_0 \sin\beta = 0
    \end{cases}
\end{equation}

即:

\begin{equation*}
    \begin{cases}
        y_0 = -x_0 \cot \beta \\
        z_0 = -x_0 \tan \alpha
    \end{cases}
\end{equation*}

不妨令$x_0 = -1$,则有:

\begin{equation}
    v_s
    = (x_0, y_0, z_0)
    = \left(
    -1,
    \cot \beta,
    \tan \alpha
    \right)
\end{equation}

因而:

\begin{equation*}
    \tan <l_s, ABEF>
    = \frac {\tan \alpha} {\sqrt{(-1)^2 + \cot ^2 \beta}}
\end{equation*}

所以,$l_s$与平面$ABEF$的夹角为

\begin{equation}
    <l_s, ABEF>
    = \arctan \left(|\sin \beta| \cdot  \tan \alpha\right)
    = \gamma
\end{equation}

\begin{mcmTheorem}{测坡角公式}
    \label{theorem:测坡角公式}
    对于一条测坡线$l_s$,$l_s$与水平面的夹角就是测坡角$\gamma$,$\gamma$与测线方向$\beta$、海底坡度$\alpha$有关,满足如下关系

    \begin{equation}
        \gamma = \arctan(|\sin\beta| \cdot \tan\alpha)
    \end{equation}
\end{mcmTheorem}

% Please add the following required packages to your document preamble:
% \usepackage{multirow}
\begin{table}[h]
    \centering
    \caption{\textbf{问题2的计算结果(保留两位小数)}}
    % 调整行距
    \renewcommand\arraystretch{1.5}
    \scalebox{0.85}{
        \begin{tabular}{cccccccccc}
            \hline
            \multicolumn{2}{c}{\multirow{2}{*}{覆盖宽度/m}} & \multicolumn{8}{c}{测量船距海域中心点处的距离/海里}                                                                         \\ \cline{3-10}
            \multicolumn{2}{c}{}                            & 0                                                   & 0.3    & 0.6    & 0.9    & 1.2    & 1.5    & 1.8    & 2.1             \\ \hline
            \multirow{8}{*}{测线方向夹角/°}                 & 0                                                   & 415.69 & 466.09 & 516.49 & 566.89 & 617.29 & 667.69 & 718.09 & 768.48 \\
                                                            & 45                                                  & 416.12 & 451.79 & 487.47 & 523.14 & 558.82 & 594.49 & 630.16 & 665.84 \\
                                                            & 90                                                  & 416.55 & 416.55 & 416.55 & 416.55 & 416.55 & 416.55 & 416.55 & 416.55 \\
                                                            & 135                                                 & 416.12 & 380.45 & 344.77 & 309.10 & 273.42 & 237.75 & 202.08 & 166.40 \\
                                                            & 180                                                 & 415.69 & 365.29 & 314.89 & 264.50 & 214.10 & 163.70 & 113.30 & 62.90  \\
                                                            & 225                                                 & 416.12 & 380.45 & 344.77 & 309.10 & 273.42 & 237.75 & 202.08 & 166.40 \\
                                                            & 270                                                 & 416.55 & 416.55 & 416.55 & 416.55 & 416.55 & 416.55 & 416.55 & 416.55 \\
                                                            & 315                                                 & 416.12 & 451.79 & 487.47 & 523.14 & 558.82 & 594.49 & 630.16 & 665.84 \\ \hline
        \end{tabular}
    }

\end{table}

\mcmSubsection{问题三:为海域设计排线方案}

\mcmSubsubsection{方案设计与证明}

\paragraph{确定测线的方向角$\beta$}

考虑一般情况,有以下简化模型:

假设有如下的直四棱柱$OBCD - O'B'C'D'$,其中$|OB| = l_{sea}$,$|OD| = w_{sea}$,$|OO'| = h$. 平面$BCEF$与平面$OBCD$之间的夹角为$\alpha$,$P$在交$O'B'$于点$A$,且$\angle B'AP = \beta$的射线上,且$|AP| = t$。需要求出代表船的点$P$在整个目标海域的条带总面积。计算过程如下:

\begin{figure}[h]
    \centering
    \includegraphics[scale=0.3]{res/img/一般情况下通式计算几何图.png}
    \caption{\href{https://www.geogebra.org/m/jzwhwcqr}{\textcolor{blue}{一般情况下通式计算几何图}}}
    \label{fig:一般情况下通式计算几何图}
\end{figure}

不难得出,坡面的解析式为:

\begin{equation}
    p: \
    (x - l_{sea})\tan \alpha + z = 0
\end{equation}

$A$点的坐标为:

\begin{equation}
    A = (l_{sea} - x, 0, h)
\end{equation}

由于单位向量$\tau_\text{测} = (\cos \beta, \sin \beta, 0)$,结合$|AP| = t$,可以得出$P$的坐标为:

\begin{equation}
    P = (l_{sea} - x + t\cos \beta,
    t\sin \beta,
    h)
\end{equation}

当前船(点$P$)的位置代入坡面解析式后,可以得出此时距离海水的深度的函数$D$的表达式为:

\begin{equation*}
    % D(\beta, x, t) = 
    % h - \left(l_{sea} - x\right)\tan \alpha
    D(\beta, t) =
    h - \left(l_{sea} - x\right)\tan \alpha
\end{equation*}

即:

\begin{equation}
    % D(\beta, x, t) = 
    % h - \left(x - t\cos \beta \right)\tan \alpha
    D(\beta, t) =
    h - \left(x - t\cos \beta \right)\tan \alpha
\end{equation}

由定理\ref{theorem:测坡角公式},根据船$P$当前的位置,可得测坡角函数$\gamma$的表达式为:

\begin{equation}
    \gamma(\beta) =
    \arctan \left(|\sin \beta| \tan \alpha\right)
\end{equation}

由定理1,根据船$P$当前的位置,可得条带宽度函数$W$的表达式为:

\begin{equation}
    % w(\beta, x, t) = 
    % D(\beta, x, t)\cdot 
    % \left(
    %     \frac{1}{\cos (\frac{\theta}{2} + \gamma(\beta))} +
    %     \frac{1}{\cos (\frac{\theta}{2} - \gamma(\beta))}
    % \right)\cdot
    % \sin {\frac{\theta}{2}}\cdot \cos \gamma(\beta)
    W(\beta, t) =
    D(\beta, t)\cdot
    \left(
    \frac{1}{\cos (\frac{\theta}{2} + \gamma(\beta))} +
    \frac{1}{\cos (\frac{\theta}{2} - \gamma(\beta))}
    \right)\cdot
    \sin {\frac{\theta}{2}}\cdot \cos \gamma(\beta)
\end{equation}

此时,对自变量$t$定积分即可求得代表船的点$P$在整个目标海域的条带总面积,即总面积$S$的函数表达式为:

\begin{equation}
    % S(\beta, x) = 
    % \int _{0} ^{R(\beta, x)} {
    %     w(\beta, x, t)dt
    % }
    S(\beta) =
    \int _{0} ^{R(\beta)} {
        W(\beta, t)dt
    }
\end{equation}

其中$R(\beta)$函数表示$t$的取值上限,且有:

\begin{equation}
    % R(\beta, x) = \min\left(
    %     \left|\frac{x - l_{sea}}{\cos \beta}\right|, 
    %     \left|\frac{w_{sea}}{\sin \beta}\right|
    % \right)
    R(\beta) =
    \begin{cases}
        \min \left \{
        \frac{x}{\cos \beta},
        \frac{w_{sea}}{\sin \beta}
        \right \}
                & \beta \in \left(0, \frac{\pi}{2} \right)   \\
        w_{sea} & \beta = \frac{\pi}{2}                      \\
        \min \left \{
        \frac{x - l_{sea}}{\cos \beta},
        \frac{w_{sea}}{\sin \beta}
        \right \}
                & \beta \in \left(\frac{\pi}{2}, \pi \right)
    \end{cases}
\end{equation}

为了让这一测线测得的条带宽度尽可能大, 考虑令条带面积的导数为零,得到使其最大的$\hat \beta$。即求解$\hat \beta$使得:

\begin{equation}
    \frac{dS(\hat \beta)}{d\beta} = 0
\end{equation}

不难得出,当$\hat \beta = \frac{\pi}{2}$时,有条带面积最大值为:

\begin{equation}
    S(\hat \beta) =
    w_{sea} \cdot (h - x\tan \alpha) \cdot \left[
        \frac{1}{\cos \left(
            \frac{\theta}{2} + \alpha
            \right)} + \frac{1}{\cos \left(
            \frac{\theta}{2} - \alpha
            \right)}
        \right]\cdot \sin \frac{\theta}{2} \cdot \cos \alpha
\end{equation}

因此,当测线的方向角$\beta = \frac{\pi}{2}$时相对优秀。

\paragraph{确定相邻测线间距$d$}

要使测量长度最短,在确定测线方向角的情况下,利用贪心的思想,尽可能控制重叠率$\eta = 10\%$。这样可以在单测线的测量长度不变的情况下,使测线条数减小,进而减小测量长度。

设第$i$条测线与第$i + 1$条测线之间的距离为$d_i$。已知海域中心的高度,可以用公式推导出第一条测线距离坡面的距离$D_1$。不妨先假设已知的是$D_1$。对于第$i$条测线,深度为$D_i$、覆盖宽度为$W_i$。根据重叠率公式、定理1和定理2不难得出以下方程组:

\begin{equation}
    \begin{cases}
        W_i       & = D_i\sin\frac{\theta}{2}\left(\frac{1}{\cos(\frac{\theta}{2}+\gamma)} + \frac{1}{\cos(\frac{\theta}{2}-\gamma)}\right) \cdot \cos \gamma       \\
        W_{i + 1} & = D_{i + 1}\sin\frac{\theta}{2}\left(\frac{1}{\cos(\frac{\theta}{2}+\gamma)} + \frac{1}{\cos(\frac{\theta}{2}-\gamma)}\right) \cdot \cos \gamma \\
        \eta      & = 1 - \frac{2d_i}{W_i + W_{i + 1}}                                                                                                              \\
        d_i       & = \frac{D_i - D_{i + 1}}{\tan \gamma}
    \end{cases}
\end{equation}

令$k, m$的值分别为:

\begin{equation}
    \begin{cases}
        k =
        \sin\frac{\theta}{2}\left(
        \frac{1}{\cos(\frac{\theta}{2}+\gamma)} +
        \frac{1}{\cos(\frac{\theta}{2}-\gamma)}
        \right) \cdot \cos \gamma \\
        m = D_i
    \end{cases}
\end{equation}

则可解得:

\begin{equation}
    \begin{cases}
        W_i = km                                                             \\
        W_{i + 1} = \left[\frac{4k}{k(1 - \eta) \tan \gamma + 2} - 1\right]m \\
        D_{i + 1} = \left[\frac{4}{k(1 - \eta) \tan \gamma + 2} - 1\right]m  \\
        d_i = \frac{2m}{\tan \gamma} - \frac{4m}{k(1-\eta) \tan^2 \gamma + 2 \tan \gamma}
    \end{cases}
\end{equation}

考虑到使用平面几何求解的困难性,接下来使用解析几何的方式求解$D_1$,如图所示:

\begin{figure}[h]
    \centering
    \includegraphics[scale=0.3]{res/img/解析式几何推导.png}
    \caption{\href{https://www.geogebra.org/m/ycxg4tw9}{\textcolor{blue}{解析式几何推导}}}
    \label{fig:解析式几何推导}
\end{figure}

计算得$D_1$用定理2中$D(0)$表示的式子为:

\begin{equation}
    D_1 = \left[ D(0) + \frac{w_{sea}}{2} \tan \alpha \right] \cdot \left( 1 - \tan \alpha \tan \frac{\theta}{2} \right)
\end{equation}

至此,结合上述公式,可使用递推得到所有测线之间的间距,即有递推公式:

\begin{align}
    D_{i} = & \begin{cases}
                  D_1                                                                    &, \text{ if } i = 1  \\
                  D_{i - 1} \cdot \left[\frac{4}{k(1 - \eta) \tan \gamma + 2} - 1\right] &, \text{ otherwise }
              \end{cases} \\
    d_i =   & \frac{2D_i}{\tan \gamma} - \frac{4D_i}{k(1-\eta) \tan^2 \gamma + 2 \tan \gamma}
\end{align}

\paragraph{方案总结}

在测线的方向角固定为$\beta = \frac{\pi}{2}$时,动态调整第$i$条与第$i + 1$条测线的间距$d_i$为以上表达式,以在保证整个海域均被覆盖的同时,控制重叠率为$\eta = 10\%$。

\mcmSubsubsection{计算流程}

\textbf{第一步: 求出第一条测线的最优方案}

若要使得整体测线方案最优,根据动态规划的最优子结构思想,需要先计算出第一条测线的最优情况,再根据第一条测线推导出其他测线方案,就可保证后续规划的每一条测线都为最优情况,最终做到整体测线方案为最优方案。

而第一条测线的最优情况,只能在边界条件下达成。分别有两种方案:

方案一 在东边界设立第一条测线.即按图4所示,AM在AB上的投影等于测线在以东向西为正方向时东西方向上的坐标。

方案二 在西边界设立第一条测线。即按图4所示,MC在AB上的投影等于测线在以西向东为正方向时东西方向上的坐标。

通过观察易知,第一条测线因无上一条测线,无需满足重叠率要求,故其覆盖宽度越大,整体探测效率越高。因此根据方案一确立第一条测线才为最优方案。

设第一条测线在以东向西为正方向时东西方向上的坐标为$x_1$(米),设海域中心在在以东向西为正方向时东西方向上的坐标为$W_c$(米),设海域中心点处的海水深度为$D_c$(米),则按图4模型可列方程:

\begin{equation}
    (W_c - x_1) \tan\alpha+D_c = \frac{x_1\sin \varphi}{\sin \frac{\theta}{2} \cos \alpha}
\end{equation}

从而解得$x_i$,进一步算得第一条测线的水深$D_1$(米)与覆盖宽度$CW_1$(米),其中$\gamma = \alpha$:

\begin{equation}
    D_1 =  (W_c - x_1) \tan{\alpha} + D_c
\end{equation}

\begin{equation}
    CW_1 = D_1\sin\frac{\theta}{2}\left(\frac{1}{\cos(\frac{\theta}{2}+\gamma)} + \frac{1}{\cos(\frac{\theta}{2}-\gamma)}\right) \cdot \cos \gamma
\end{equation}

\textbf{第二步: 进一步推导其他测线方案}

得到第一条测线的位置与覆盖宽度后,便可推出下一条测线的最优情况。

问题三要求所规划的整体测量长度最短,且相邻条带之间的重叠率满足 10\%$\sim$20\% 的要求。故意味着下一条最优测线与上一条最优测线的重叠率为10\%。设第$i$条测线在以东向西为正方向时东西方向上的坐标为$x_i$(米),水深为$D_i$(米),覆盖宽度为$CW_i$(米),故根据式(8)可列方程如下:

\begin{equation}
    1-\frac{2(x_i-x_{i-1})}{CW_{i-1}+CW_i} = 0.1
\end{equation}

其中,$CW_i$与$D_i$的计算公式如下:

\begin{equation}
    CW_i = D_i\sin\frac{\theta}{2}\left(\frac{1}{\cos(\frac{\theta}{2}+\gamma)} + \frac{1}{\cos(\frac{\theta}{2}-\gamma)}\right) \cdot \cos \gamma
\end{equation}

\begin{equation}
    D_i =  (W_c - x) \tan{\alpha} + D_c
\end{equation}

联立式(34)、式(35)、式(36),最终可解得$x_i$的值,即可确定下一条测线的最优方案。

循环执行第二步,并设立边界条件$ x \leq 2W_c $,即可求出一组测量长度最短、可完全覆盖整个待测海域的测线。

\mcmSubsubsection{计算结果}


\begin{figure}[h]   
    \centering
    \includegraphics[scale=0.4]{res/img/问题三示意图.png}
    \caption{\href{https://www.geogebra.org/classic/bvejtecw}{\textcolor{blue}{问题三求解示意图}}}
    \label{fig:问题三求解示意图}
\end{figure}

如图10所示,红色线为满足整体测量长度最短,且相邻条带之间的重叠率满足 10\%$\sim$20\% 的测线方案的前两条测线(其余未画出)。若用测线在$x$轴上的坐标来表示测线位置,则该测线方案下的测线位置如表3所示:

\begin{table}[]
    \centering
    \caption{\textbf{问题3的计算结果}}
    % 调整行距
    \renewcommand\arraystretch{1.5}
    \scalebox{0.6}{
        \begin{tabular}{cccccccc}
            \hline
            测线标号 & 测线所在位置/m    & 测线标号 & 测线所在位置/m & 测线标号 & 测线所在位置/m & 测线标号 & 测线所在位置/m \\ \hline
            1    & 343.2623502 & 11   & 4566.553 & 21   & 6431.147 & 31   & 7254.242 \\
            2    & 936.3682148 & 12   & 4828.662 & 22   & 6546.851 & 32   & 7305.317 \\
            3    & 1483.515013 & 13   & 5070.191 & 23   & 6653.47  & 33   & 7352.382 \\
            4    & 1987.700552 & 14   & 5292.756 & 24   & 6751.717 & 34   & 7395.752 \\
            5    & 2452.298095 & 15   & 5497.845 & 25   & 6842.251 &      &          \\
            6    & 2880.416039 & 16   & 5686.83  & 26   & 6925.675 &      &          \\
            7    & 3274.918716 & 17   & 5860.977 & 27   & 7002.549 &      &          \\
            8    & 3638.445553 & 18   & 6021.45  & 28   & 7073.387 &      &          \\
            9    & 3973.428732 & 19   & 6169.322 & 29   & 7138.663 &      &          \\
            10   & 4282.109466 & 20   & 6305.584 & 30   & 7198.814 &      &          \\ \hline
            \end{tabular}
    }
\end{table}

在该测线方案下,当满足模型假设情况时,该测线方案可完全覆盖所划海域,并可做到整体测量长度最短。测量长度为:132,988.489646273916米。

\mcmSubsection{问题四:在已有数据情况下的测线方案设计}

在问题四中,定义重叠率为:

\begin{equation}
    \eta = \frac{1}{2} \left( \frac{\Delta S}{S_i} + \frac{\Delta S}{S_{i+1}} \right)
\end{equation}

\begin{figure}[h]
    \centering
    \includegraphics[scale=0.3]{res/img/第四问解答的算法流程图.png}
    \caption[short]{第四问解答的算法流程图}
\end{figure}

\mcmSubsubsection{第一步: 求出最优的测线方向角}

根据问题三的证明,为了尽可能保证每条测线的重叠率最小,同时航行路程最短,应该尽可能让航行条带等宽。因此,在探测平坦的海底坡面时,只要尽力保证测线上每个点的深度不变即可。而为衡量该不变程度,接下来定义海底坡面的平坦率$\varepsilon$。

首先定义某条测线的平坦率。在某条测线的海域下,从起点A到终点B之间的海底深度$D$的方差称为线段AB在该海域上的的平坦率。而对于问题四中给出的离散数据而言,应找到线段AB附近位置的海底深度数据,依次近似求出该线段的平坦率。定义该数据列表为$D_{i}$,平坦率$\varepsilon$满足:

\begin{equation}
    \varepsilon = \frac{1}{n} \sum^{n}_{i=1}(D_i - \overline{D} )^2
\end{equation}

其中,$\overline{D}$为深度的平均值。

接下来,枚举出全部方向上的平坦率,得到最优解,从而得到一个测线方案。

随后,以(0,0)为初始位置,令$\beta = \frac{k}{72}\pi$,其中$k = 0, 1, \cdots, 36$,求出在该方向下的平坦率即可。

% /TODO 表呢??????????

\mcmSubsubsection{第二步:求出最优测线方案}

在上述所有的方向中,平坦率最大值所对应的方向$\hat \beta$,认为是测线方向最合适的。随后,以该方向为基础,每间隔$d$增加一条测线,延伸到海域外围为止。随后,计算所有测线的平坦率。若有$n$条测线,对于第$i$条测线,其平坦率设为$\varepsilon_i$,并且所有测线平坦率对应的平均值,视为整个海域的平坦率,则有:

\begin{equation}
    \overline{\varepsilon} = \frac{1}{n} \sum^{n}_{i=1}\varepsilon_i
\end{equation}

\mcmSubsubsection{第三步:求出最优测线方案下的各种指标}

首先,根据三维线性规划,求出某条测线能够覆盖的点的集合。

其次,将点集映射到水平平面后,根据安德鲁算法(凸包算法)求出该点集的轮廓,进而求出面积。同时与相邻测线的点集再取交集,以求出该交集的凸包的面积,为之后求重叠率做准备。

最后,根据前面的计算过程,得出重叠率。

对于平面上的一些点要求它们的凸包,需要首先将它们以横坐标作为主关键字、纵坐标为次关键字进行排序,然后先依次从左到右遍历点找到下凸包,再依次从右到左遍历点找到上凸包,并选择性地将一些点放入栈中。如果栈里的点数超过2个,并且最近入栈的两个点形成的向量与倒数第二个点与当前节点形成的向量的叉积不小于0,则将栈中的点一直出栈,直到不满足要求。此时再将当前遍历的点入栈,完成一个点的遍历。当遍历结束后,栈中剩余的所有点的集合构成了凸包。

对于计算凸多边形的面积,使用向量的叉乘。假设一个凸多边形的顶点数为$n$,决定一个顶点后,可以用$n-3$条经过该点的直线将该凸多边形分成$n-3$个小三角形。对于每一个三角形,直接使用叉乘计算对应的面积即可。

接下来,在给定测线经过点$P(x_1, y_1, k)$和点$Q(x_2, y_2, k)$的情况下,计算条带覆盖面积。容易得:随着船只沿着测线方向行进,多波形成一对平面,如下图所示:

\begin{figure}[htbp]
    \centering
    \begin{subfigure}[b]{0.45\textwidth}
        \centering
        \includegraphics[scale=0.15]{res/img/覆盖域平面_几何.png}
        \caption{覆盖域平面立体图}
        \label{fig:覆盖域平面立体图}
    \end{subfigure}
    \hfill
    \begin{subfigure}[b]{0.45\textwidth}
        \centering
        \includegraphics[scale=0.15]{res/img/覆盖域平面_截面正视图.png}
        \caption{覆盖域截面正视图}
        \label{fig:覆盖域截面正视图}
    \end{subfigure}
    \caption{\href{https://www.geogebra.org/m/zafwcq6e}{\textcolor{blue}{覆盖域截面图}}}
    \label{fig:覆盖域截面图}
\end{figure}

因此,可以通过解下列方程组得到定理4:

\begin{equation}
    \begin{cases}
        a(x_1 - x_2) + b(y_1 - y_2) = 0 \\
        \tan^2\frac{\theta}{2} = \frac{c^2}{a^2 + b^2}
    \end{cases}
\end{equation}

\begin{mcmTheorem}{覆盖域平面方程}
    对于三维坐标系中的P$(x_1, y_1, k)$, Q$(x_2, y_2, k)$,PQ为测线,那么该测线覆盖域平面的法向量$(a, b, c)$为:

    \begin{equation}
        \begin{cases}
            a = \frac{y_2-y_1}{x_1-x_2} \\
            b = 1                       \\
            c = \pm \sqrt{\tan^2\frac{\theta}{2} \cdot (a^2+1)}
        \end{cases}
    \end{equation}

    令$c \geq 0$,则平面方程分别为:

    \begin{equation}
        \begin{cases}
            ax + by + cz - ax_1 - by_1 - ck = 0 \\
            ax + by - cz - ax_1 - by_1 + ck = 0
        \end{cases}
    \end{equation}
\end{mcmTheorem}

随后,附件里的值可以被绘制成三维空间下的离散点。取所有深度低于直线$PQ$的高度,且介于两平面之间的所有离散点作为点集,求解其凸包对应的面积,即为当前测线下的覆盖区域面积$S$。其俯视示意图如下所示。

\begin{figure}[h]
    \centering
    \includegraphics[scale=0.3]{res/img/俯视平面夹凸包示意图.png}
    \caption{\href{https://www.geogebra.org/m/vndf5wzp}{\textcolor{blue}{俯视平面夹凸包示意图}}}
    \label{fig:俯视平面夹凸包示意图}
\end{figure}

与此同时,条带之间还会重叠,导致相邻两个条带的凸包有重合部分,如图所示:

\begin{figure}[h]
    \centering
    \includegraphics[scale=0.3]{res/img/俯视凸包重叠示意图.png}
    \caption{\href{https://www.geogebra.org/m/sf4pvazp}{\textcolor{blue}{俯视凸包重叠示意图}}}
    \label{fig:俯视凸包重叠示意图}
\end{figure}

记第一个条带上点的凸包面积为$S_1$,第二个条带上点的凸包面积为$S_2$,两个条带重合部分上的凸包面积为$S_r$。定义相邻两个条带的重合率$\eta$为:

\begin{equation}
    \eta = 1 - \frac{2 S_r}{S_1 + S_2}
\end{equation}

在不超过当前海域的情况下,往垂直于原测线且平行于水平面方向每隔等距离$d$为添加的测线,即可计算出所有测线对应的覆盖区域面积、相邻条带之间的重叠率$\eta$和测线的总长度。

由程序计算,得到较优的测线的方向角$\beta \approx 26.88 ^{\circ} $。据此,可以用公式求解出最优的等距测线间隔$d$,进而求出想要的结果。

% ==================================================
% @brief    模型的评价与改进
% ==================================================

\quad\newline
\mcmSection{模型的评价与改进}

\mcmSubsection{本文模型的优点}

\begin{enumerate}
    \item 本文模型使用简单的几何图形描述了问题,易于读者理解。并可在已知少量参数的情况下近似拟合较为高效的测线方案。
    \item 本文的解题代码已尽可能封装,更具普适性,可简单调整参数后探究其他情况下的结果。
    \item 本文四问解题逻辑逐层递进,推导具有合理性。
\end{enumerate}

\mcmSubsection{本文模型的缺点}

\begin{enumerate}
    \item 模型忽略了大量实际测探中的情况,如地形、海底地质、海水、船体构造等对测探范围造成的影响。应对情况复杂的海域时,本模型适应性较差。
    \item 问题三仅考虑了所有测线的理论覆盖宽度对指定区域全覆盖,并未考虑外缘波束容易丢失,会造成部分区域探测不到的情况。故不一定能完美解决实际问题,仅能证明为理论最佳方案。
\end{enumerate}

\mcmSubsection{本文模型的改进}

\begin{enumerate}
    \item 需进一步测量水域的客观环境,海底地质,海底地形的多种因素设计测线。
    \item 需进一步探究是否有其他条件影响多波束测量的覆盖宽度。
    \item 可进一步优化程序代码,以在输出测线方案同时输出预览图,便于进一步实地勘测记录,进行细节调整。
\end{enumerate}