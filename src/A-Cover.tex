\thispagestyle{empty}   % 定义起始页的页眉页脚格式为 empty —— 空,也就没有页眉页脚

\begin{center}
    \textbf{\fontsize{20}{1.5}关于对海域地形探测方案设计的研究}

    \textbf{摘 要}
\end{center}





% ==================================================
% @brief    摘要
% ==================================================

多波束测探系统是海底测探领域中一项重要的技术,通过利用声波的传播特性,可以用来测绘海底坡面的地形图。而海底坡面地形变化差异大,又因为声波的传播的物理特性,\textbf{多波束测探系统的覆盖区域会随着探测船所在位置的海底深度的变化而变化}。在实际探测过程中,该特性将导致浅海海域的海床无法被探测,而在深海海域的海床探测区域重叠度过高,前者精度低,后者效率低。为了提高精度和效率,本文做出相应的研究。

对于\textbf{问题一},根据对现实情景的分析,将实际问题抽象为几何问题。利用\textbf{三角函数}与\textbf{正弦定理}列出相关方程,最终得到覆盖宽度公式(定理一)与海底深度函数(定理二)。此时,多波束测探的覆盖宽度$W$与相邻条带之间重叠率$\eta$的数学模型将得以建立。最后,通过使用$Python$语言编程后(源码\textbf{question1.py}),输入对应的数据,可得出结果(附件\textbf{result1.xlsx})。

对于\textbf{问题二},相比于问题一,从二维平面拓展到三维空间。同样对该情景的分析,将实际问题抽象为数学问题。根据解析几何的理论,建立适当的坐标系,\textbf{求出海底坡面和测坡面的法向量},即可求出测坡角(定理三)。最后将三维空间转化到了二维平面上,使用定理一,即可建立多波束测探覆盖宽度的数学模型。同样使用$Python$语言编程(源码\textbf{question2.py}),输入对应的数据后,可得出问题二结果(附件\textbf{result2.xlsx})。

对于\textbf{问题三},首先抽象出问题三的几何情景,建立适当的空间直角坐标系,可得条带宽度函数$W(\beta,t)$,则条带宽度的总面积则为宽带函数的含参积分上限函数。分析该定积分,可证得\textbf{沿着正南正北方向设计测线方案比沿着其他方向设计测线方案效率更高}。以该结论为基础,使用贪心算法编程(源码\textbf{question3.py})后可得排线方案(附件\textbf{result3.xlsx}),在该方案中,总测量长度为$132988.49m$。

对于\textbf{问题四},首先根据已有的数据,可绘制出海床的大致地形图。根据问题三中的结论,为了保证一条探测线的效率最高,应该尽可能地保证海底深度不变。因此本文定义指标\textbf{平坦率$\varepsilon$用来衡量在某个方向下海底深度变化趋势},通过程序计算(源码\textbf{question4.py})可得,当测线方向角约为$26.88^\circ$时,平坦率$\varepsilon=188.34$,此时,海底相对最平坦。有了最优测线方向角以后,将开始具体的测线方案,为了求覆盖宽度,分别使用广度优先搜索出覆盖区域,通过凸包算法求出覆盖面积与覆盖率。编程后可得计算结果,在该方案下得出计算结果(附件\textbf{result4.xlsx})。

\quad\newline
\newline
\textbf{关键词}:多波束测探系统 \quad 解析几何 \quad 凸包 \quad 三维线性规划 \quad 广度优先搜索
