% =======================================
% 参考文献
% =======================================

\bibliography{src/E-Reference}

% 引用所有 E-Reference.bib 里面的全部参考文献,不论在论文中是否被引用
\nocite{*}


\appendix
\section{主要使用的软件}

\begin{enumerate}
    \item 文字编辑方案:Visual Studio Code + \LaTeX + Git + Zotero
    \item 程序模拟:PyCharm + Python
    \item 绘图软件:XMind + PyCharm + Python + GeoGebra
\end{enumerate}

\section{程序代码}

\begin{lstlisting}[caption={类的定义语句}]
    #include<iostream>
    using namespace std;

    int main(){
        cout << "Hello, 数学建模" << endl;
        return 0;
    }
\end{lstlisting}

\section{GeoGebra绘制链接}

\begin{enumerate}
    \item \href{https://www.geogebra.org/m/hpkkarys}{\textcolor{blue}{问题一求解图}}
    \item \href{https://www.geogebra.org/m/f6kfjvru}{\textcolor{blue}{问题二求解图}}
    \item \href{https://www.geogebra.org/m/n8saurfn}{\textcolor{blue}{测坡角与测线方向角的函数图像}}
    \item \href{https://www.geogebra.org/m/xcvstdzg}{\textcolor{blue}{理想情况下的覆盖区域几何图}}
    \item \href{https://www.geogebra.org/m/absuxwpk}{\textcolor{blue}{一般情况下的覆盖区域几何图}}
\end{enumerate}
