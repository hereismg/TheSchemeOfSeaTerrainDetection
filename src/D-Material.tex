% =======================================
% 参考文献
% =======================================

\bibliography{src/E-Reference}
\begin{verbatim}
[1] 测线[OB]. 百度百科. https://baike.baidu.com/item/%E6%B5%8B%E7%BA%BF0\

2601563?fr=ge_ala.

[2] 夏伟,刘雁春,边刚,等.基于海底地貌表示法确定主测深线间隔和测图比例尺[C]

//第十五届海洋测绘综合性学术研讨会.0[2023-09-10].

[3] 凸包[OB]. 百度百科. https://baike.baidu.com/item/%E5%87%B8%E5%8C%85

?fromModule=lemma_search-box

[4] 李伟,赖陪伟.影响多波束测量精度与质量的因素分析与措施[C]

//中国航海学会航标专业委员会测学组学术研讨会.2008.

\end{verbatim}

% \nocite{*}

% =======================================
% 附录
% =======================================

\appendix
\section{主要使用的软件}

\begin{enumerate}
    \item 文字编辑方案:Visual Studio Code + \LaTeX + Git + Zotero
    \item 程序模拟:PyCharm + Python
    \item 绘图软件:XMind + PyCharm + Python + GeoGebra
\end{enumerate}

\section{程序代码}

\begin{lstlisting}[caption={question1.py}]
from base import *

distance = [-800, -600, -400, -200, 0, 200, 400, 600, 800]
spacing = [distance[i] - distance[i - 1] for i in range(1, len(distance))]
theta = (120 / 180) * math.pi
alpha = (1.5 / 180) * math.pi
DEPTH_CENTER = 70

result_depth_list = [calculate_depth(i, alpha, DEPTH_CENTER) for i in distance]
# 第一问中,gamma = alpha
result_cover_width_list = [calculate_cover_width(theta, alpha, i) for i in result_depth_list]
result_overlap_rate_list = \
    [calculate_overlap_rate(spacing[i - 1], result_cover_width_list[i - 1] + result_cover_width_list[i])
        for i in range(1, len(result_cover_width_list))
        ]

print("测线距中心点处的距离(米):")
print(distance)
print("海水深度(米):")
print(result_depth_list)
print("覆盖宽度(米):")
print(result_cover_width_list)
print("与前一条测线的重叠率(%):")
print(result_overlap_rate_list)
\end{lstlisting}

\begin{lstlisting}[caption={question2.py}]
import math

import numpy as np

from base import *

# 测量船距海域中心点处的距离(米)
milage = [0 * 1852, 0.3 * 1852, 0.6 * 1852, 0.9 * 1852, 1.2 * 1852, 1.5 * 1852, 1.8 * 1852, 2.1 * 1852]
# 测线方向夹角(角度制)
beta = [0, 45, 90, 135, 180, 225, 270, 315]
# 多波束换能器的开角(弧度制)
theta = (120 / 180) * math.pi
# 海底坡面坡度(弧度制)
alpha = (1.5 / 180) * math.pi
# 测坡角(弧度制)
gamma = [math.atan(abs(math.sin(math.radians(i))) * math.tan(alpha)) for i in beta]
# 海域中心点处的海水深度(米)
DEPTH_CENTER = 120

for i in range(0, len(beta)):
    # 水深(米)
    depth = [calculate_depth(- j * math.cos(math.radians(beta[i])), alpha, DEPTH_CENTER) for j in milage]
    # 覆盖宽度计算结果(米)
    result_cover_width_list = [calculate_cover_width(theta, gamma[i], depth[j]) for j in range(0, len(depth))]
    print("测线方向夹角为 " + str(beta[i]) + "° 时,覆盖宽度(米)为:")
    print(result_cover_width_list)
    # 方差
    # print(np.var(result_cover_width_list))
    # 均值
    # print(np.mean(result_cover_width_list))    
\end{lstlisting}

\begin{lstlisting}[caption={question3.py}]
import math
from sympy import symbols, Eq, solve

from base import *


def solution_question3(WIDTH, alpha, beta, theta, DEPTH_CENTER):
    """
    问题三的解题函数
    :param WIDTH: 东西宽度(米)
    :param alpha: 坡面坡度(弧度制)
    :param beta: 测线方向夹角(弧度制)
    :param theta: 多波束换能器的开角(弧度制)
    :param DEPTH_CENTER: 海域中心点处的海水深度(米)
    :return: 测线方案:以东到西为正方向,每条测线的东西方向坐标(米)
    """
    # 海域中心点处的东西方向坐标(米)
    WIDTH_CENTER = WIDTH / 2
    # 测坡角(弧度制)
    gamma = math.atan(abs(math.sin(math.radians(beta))) * math.tan(alpha))
    phi = (math.pi - theta) / 2 - math.atan(abs(math.sin(math.radians(beta))) * math.tan(alpha))
    # 航线相对于左边界的距离(米)
    x = 0
    # 设测线方案解集为x_list,用于储存以东到西为正方向,每条测线的东西方向坐标(米)
    x_list = []
    # 初始化:计算第一条测线的位置,设为last_x。之后last_x表示为前一条测线的位置。
    last_x = symbols("last_x")
    equation = Eq((WIDTH_CENTER - last_x) * math.tan(alpha) + DEPTH_CENTER,
                    math.sin(phi) * last_x / math.sin(theta / 2) / math.cos(alpha))
    last_x = solve(equation, last_x)[0]
    x_list.append(last_x)
    # 初始化:计算出第一条测线的覆盖宽度,设为cover_width1。之后cover_width1表示为前一条测线的覆盖宽度。
    depth = calculate_depth(last_x - WIDTH_CENTER, alpha, DEPTH_CENTER)
    cover_width1 = calculate_cover_width(theta, gamma, depth)

    # 开始计算每一条测线的位置,其中测线位置不得超过所划海域范围,即x不得大于WIDTH
    while x <= WIDTH:
        # 设 x 为未知数
        x = symbols("x")
        # 根据测线位置得在东西方向上测线距离海域中心点的距离(米),从而计算出测线位置的水深(米)。
        depth = calculate_depth(x - WIDTH_CENTER, alpha, DEPTH_CENTER)
        # 根据测线位置的水深(米)得到该条测线的覆盖宽度(米)
        cover_width2 = calculate_cover_width(theta, gamma, depth)
        # 为满足测量长度最短,且相邻条带之间的重叠率在10%~20%之间,相邻两条测线的重叠率需为10%
        equation = Eq(1 - 2 * (x - last_x) / (cover_width1 + cover_width2), 0.1)
        # 求解得到 x
        x = solve(equation, x)[0]
        # 再次判定,如果 x 在海域范围内,则放入解集中,反之不纳入
        if x <= WIDTH:
            x_list.append(x)
        # 将当前规划的测线位置定为上一条测线的位置,准备规划下一条测线
        last_x = x
        # 将当前规划的测线的覆盖宽度定为上一条测线的覆盖宽度,准备规划下一条测线
        cover_width1 = calculate_cover_width(theta, alpha, calculate_depth(x - WIDTH_CENTER, alpha, DEPTH_CENTER))
    # 计算完成,返回解集
    return x_list


if __name__ == '__main__':
    LENGTH = 2 * 1852
    WIDTH = 4 * 1852
    alpha = math.radians(1.5)
    theta = math.radians(120)
    beta = math.radians(90)
    DEPTH_CENTER = 110

    x_list = solution_question3(WIDTH, alpha, beta, theta, DEPTH_CENTER)
    print("测线方案:以东到西为正方向,测线的坐标(米)如下:")
    print(x_list)
    print("相邻测线间的距离(米):")
    print([x_list[i] - x_list[i - 1] for i in range(1, len(x_list))])
    print("整个测线方案的测量长度(米):")
    print(LENGTH * len(x_list))
\end{lstlisting}

\section{GeoGebra绘制链接}

\begin{enumerate}
    \item \href{https://www.geogebra.org/m/hpkkarys}{\textcolor{blue}{问题一求解图}}
    \item \href{https://www.geogebra.org/m/f6kfjvru}{\textcolor{blue}{问题二求解图}}
\end{enumerate}
